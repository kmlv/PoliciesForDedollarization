The mass adoption of a foreign currency (e.g. currency dollarization) often occurs amid economic crises in  small economies. 
Interestingly, after the economy has returned to macroeconomic stability the dollarization persists even . 
This is particularly relevant to central banks as dollarization can limit its policy effectiveness; yet, there is insufficient research on the relative impact of de-dollarization policies.
We extend existing theory (based on \cite{RePEc:oup:restud:v:60:y:1993:i:2:p:283-307.}) and device a laboratory experiment to study the impact of different policies on the degree of currency dollarization, measured by the acceptance rate of foreign currency. 
Our experiment explores the impact of the following policies or environmental factors:
(1) A tax on domestic transactions in foreign currency;
(2) A tax on holdings of foreign currency;
(3) Information on the likelihood of acceptance of the foreign currency in the local economy.
In this version we present the results of a pilot study.


%%%%%%%%%%%%%%%%%%%%%%%%%%%%

